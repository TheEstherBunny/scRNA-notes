\documentclass[a4paper,12pt]{article}

\title{scRNA-seq notes}
\author{Esthy Hung} 
\date{\today}


\begin{document}
\maketitle

Steps of single-celll RNA-seq analysis include pre-processing (quality control, normalisation, data correction, feature selection, and dimensionality reduction) and cell-and gene-level downstream analysis.
\section{Workflow}
\subsection{Pre-processing}
\begin{enumerate}
\item Raw data processing
\item Count matrices
\item Visualisation
\begin{itemize}
\item Quality control
\item Normalisation
\item Data Correction
\item Feature selection
\end{itemize}
\end{enumerate}

\subsection{Downstream Analysis}
\begin{enumerate}
\item clustering
\item marker identification
\item trajectory inference
\item Differential expression
\item gene dynamics
\item compositional analysis
\item metastable states
\end{enumerate}

\section{Pre-Processing}
Raw data generated by sequencing machines are processed to obtain matrices of molecular counts (count matrices) or, alternatively, read counts (read matrices), depending on whether unique molecular identifiers (UMIs) were incorporated in single-cell libray construction protocol.
\paragraph{}
Raw data processing take care of read quality control, assigning reads to their cellular barcodes and mRNA molecules of origin, genome alignment, and quantification. 
Read and count matrices have dimensions (no. barcodes, no. transcripts).


\paragraph{Quality control}
\paragraph{}
Three QC Covariates:
\begin{itemize}
\item Count depth: no. counts per barcode
\item No. genes per barcode
\item Fraction of counts from mitochondrial genes per barcode
\end{itemize}

Considering QC covariates in isoation can lead to misinterpretation of cellular signals so all three should be considered jointly when univariate thresholding decisions are made. In future, filtering models that account for multivariate QC dependencies may provide more sensitive QC options.
Raw count matrices can include over 20,000 genes. This can be filtered out by filtering out genes not expressed in more than a few cells and thus not informative of cell hetrogeneity.
To summarise: Perform QC by finding outlier peaks in the number of genes, the count depth and the fraction of mitochondrial reads. Consider covariates jointly instead of separately.

\subsection{Normalisation}
\paragraph{}
Each count in a count matrix represent successful capture, reverse transcription and sequencing of a molecule of cellular mRNA.
Normalisation scales count data to obtain correct relative gene expression abundances between cells.
\paragraph{}
Most commonly used normalisation protocol is count depth scaling - 'counts per million'
\paragraph{summary}
\begin{itemize}
\item Scran is recommended for normalisation of non-full-length datasets An alternative is to evaluate normalisation approache via scone especially for plate-based datasets.  Full-length scRNA-seq protocols can be corrected for gene length using bulk methods.
\item There is no consensus on scaling genes to 0 mean and unit variance. We prefer not to scale gene expression.
\item Normalised data should be log(x+1) -transformed for use with downstream analysis methods that assume data are normally distributed.
\end{itemize}

\subsection{Data correction and integration}
Normalised data may still contain unwanted variability. Data correction targets further technical and biological covariates like batch, dropout, or cell cycle effect.
We consider correction of these effects separately.

\paragraph{Biological effects}
Most common correction- remove effect of cell cycle on transcriptome.  Can be performed by a simple linear regression aggainst a cell cycle score.
Removing cell cycle effects can improve inference of developmental trajectories but cell cycle signals can also be informative of the biology. EG proliferating cell populations can be identified based on cell cycle scores.
\paragraph{Technical effects}
Most prominent covariates are count depth and batch.
Technical count effects may remain after normalisation as no scaling method can infer expression values of genes not detected due to poor sampling. 
Regressing out ount depth effects can improve erformance of trajectory inference algorithms, which rely on finding transitions between cells.

\paragraph{Batch effects and data integration}
Batch effects can occur when cells are handled in dinstinct groups; eg cells on different chips, different sequencing lanes or harvested at different times.
\subparagraph{Batch correction}
Correcting for batch effects between samples or cells in the same experiment from bulk RNA-seq. Typically use linear methods.
\subparagraph{Data integration}
Distinguish the batch correction from the integration of data from multiple experiements. Typically use non-linear approaches.


\paragraph{Expression recovery}
Measurements contain various sources of noise, a particular aspect of this being dropout. Expression recovery has been shown to improve esitmation of gene-gene correlations.

\paragraph{summary}
\begin{itemize}
\item Regress out biological covariates for trajectory inference andif other biological processes of interest are not masked dby the regressed out biological covariate.
\item Regress out technical and biological covariates jointly rather than serially
\item Plate-based dataset pre-processing may require regressing out ccounts, normalisation via non-linear normalisation methods or downsampling.
\item Data integration and batch correction should be performed by different methods. Data integration toold may over-correct simple batch effects
\item be careful of signals found only after expression recovery
\end{itemize}

\subsection{Feature selection, dimensionality reduction and visualisation}
A single-cell RNA-seq dataset can contain expression values for up to 25,000 genes. Many genes will mostly contain zero-counts, many will not be informative. Even after filtering out zero-count genes in QC, feature space can have over 15,000 dimensions.  So it's nice to reduce dimensionaity of dataset with other tools.

\paragraph{Feature selection}
Processing of filtering out and keeping genes that are 'informative' of variability in the data. Highly variable genes (HGVs) are usually used. Genes with highest variance-to-mean ratio are selected as HGVs in each bin. This is preferably done after technical data correction.

\paragraph{Dimensionality Reduction}
The biological manifold on which cellular expressions lie can be sufficiently described by far fewer dimensions than the number of genes; dimensionality reduction aims to find these dimensions.
There are two main objectives of dimensionality reduction:
\paragraph{Visualisation} - attempt to optimally describe dataset in 2/3 dimensions. These used as coordinates on a scatter plot to obtain a visual representation of the data.
\paragraph{Summarisation}- method of reducing data to its essential components by finding inherent dimensionality of the data - helpful for downstream analysis.
\paragraph{}
Reduced dimensions are generated with linear or non-linear gene expression vectors. Interpretability (especially in non-linear case) of reduced dimensions is sacrificed in the process.
Two popular dimensionality reduction techniques that are principally summarisation methods are: Principal Component Analysis (PCA),  and diffusion maps.
\paragraph{PCA} - A linear approach, generates reduced dimensions by maximising the captured residual variance in each further dimension.  It is the basis of many currently available analysis tools for clustering or trajectory inference.
\begin{itemize}
\item Summarises  data set via its top N principal components, where N can be determined by eg 'elbow' heuristics.
\item Advantage: Distances in reduced dimensional space have a consistent interpretation in all regions of this space - thus can correlate quantities of interest with principal components to assess their importance.
\end{itemize}
\paragraph{Diffusion maps} - Non-linear data summarisation technique
\begin{itemize} 
\item As diffusion components emphasise transitions in the data, they are principally used when continuous processes such as differentiation are of interest.
\item Typically, each diffusion component i.e. diffusion map component highlights the heterogeneity of a different cell population
\end{itemize}
\paragraph{Visualisation}

\paragraph{summary}
\begin{itemize}
\item Select between 1000 and 5000 HGVs depending on complexity of dataset
\item Feature selection methods using gene expression means and variances cannot be used when gene expression values have been normalised to zero mean and unit variance, or when residuals from model fitting are used as normalised expression values. Thus, consideration of which pre-processing to perform is important before selecting HGVs.
\item Dimensionality methods should be considered separately for summarisation and visualisation.




\end{itemize}•

\subsection{Stages of pre-processsed data}

\paragraph{summary}

\section{Downstream analysis}
Performed after pre-processing. We obtain descriptions of the underlying biological system described by the data by fitting interpretable models. 
\paragraph{Examples:}
\begin{itemize}
\item groups of cells with similar gene expression profiles representing cell-type clusters
\item small changes in gene expression between similar cells denoting continuous (differentiation) trajectories
\item genes with correlated expression profiles indicating co-regulation
\end{itemize}

Downstream analysis -can be divided into cell-level or gene-level approaches. 
\subsection{Cluster Analysis}
The attempt to explain the heterogeneity in the data based on a categorisation of cells into groups.
\paragraph{Clustering}
Organising cells into clusters help us to infer the identity of member cells.
Expression profile similarity is determined via distance metrics, which often take dimensionality-reduced representations as input.
\paragraph{Clustering} - a classical unsupervised machine learning problem based directly on a distance matrix.
\paragraph{Community detection methods} - grapth-partitioning algorithms and thus rely on a graph representation of single cell data. Often faster than clustering.
A standard approach has become mulit-resolution modularity optimisation. 

\paragraph{Cluster annotation}
On gene-level we can find gene signatures of each cluster by analysing clustered dat, labelling with a 'marker gene'.
\paragraph{summary}
\begin{itemize}
\item
\end{itemize}

\paragraph{Compositional Analysis}
At cell level, we can analyse clustered data in terms of compositional structure - looking at the proportion of cells that fall into each cell-identity cluster. These proportions can change in response to disease.
Statistical tests over changes in the proportion of a cell-identity cluster between samples are dependent on one another.

\subsection{Trajectory analysis}
We regard data as a snapshot of  dynamic process and investigate the underlying process. 
\paragraph{Trajectory inference}
Required in order to capture transitions between cell identities, branching differentiation processes, or gradual, unsynchronised changes in biological function, using dynamic models of gene expression.
\paragraph{}
We reconstruct the underlying processs by finding paths through cellular space that minimise transcriptional changes between neighbouring cells. Pseudotime variable describes the ordering of cells along paths
\paragraph{} Examples of current models:
\begin{itemize}
\item Simple linear trajectories
\item Sime bifurcating trajectories
\item Complex graphs, trees or multifurcating trajectories
\end{itemize}
No individual method performs optimally for all trajectories.
\paragraph{Slingshot} - Concluded to be the best for simple trajectories from linear to bi- and multi-furcating models.
\paragraph{PAGA} - recommended for more complex trajectories.
\paragraph{} TI is typically performed after clustering. 

\paragraph{Gene expression dynamics} - Genes that vary smoothly across pseudotime characterise the trajectory and can be used to identify the underlying biological process. This group of trajectory-associated genes is expected to contain the genes that regulate the modelled process. Regulator genes help us understand how and why biological processes are triggered and represent potential drug targets.



















































\end{document}
